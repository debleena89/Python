\section{Related Work} \label{sec2}
\noindent
%Replacement attacks \cite{GhoshHD12} carried out by internal employees, phishing attacks, code injection attacks have been known to be a vector for cyber-attacks for a while. Stuxnet analysis showed that such
%attacks were part of the repertoire in that case.  The work in \cite{DBLP:conf/acns/MingXLW0M15} %\cite{DBLP:journals/virology/MingXLWLM17} 
%describes the idea of replacement attacks on malware behaviour. Malware belonging to the same family are supposed to show similar behaviour but if an attack changes the behaviour specification, then inspite of being from same family, their behaviour will be different. Malware
%  interact with operating system through system call. A system call dependency graph(SCDG) shows the data flow dependency among system call. In this paper,
%  they propose the idea of SCDG replacement and the after replacement effect on the malware behaviour. To detect the attack, malware analysts are required 
%  to put extra efforts to re-analyse the similar samples. Further,  
The idea of automata based control scheduling has been discussed in \cite{WeissA07}. They express
the communication interface among the control components using a formal language. This work illustrates
how the interfaces of discrete-time switched linear systems
can be expressed using a B\"{u}chi automaton. Authors in \cite{AlurW08}  describe CPU scheduling
in terms of finite machines over infinite words. The infinite words depict the particular time 
slot for which a particular resource can be allotted to a specific control component. 
The work in \cite{WeissFAA09} applies automata based scheduling on networks, to formalize
the effect of bus scheduling and network stability. Further, \cite{GhoshMDHD16} describes
how the interfacing among 
the control actions can be modeled by a B\"{u}chi automaton to guarantee stable and reliable
scheduling. This work presents a methodology for construction of a scheduler automaton
where each state represents one particular control schedule while generating the permissible 
schedule in terms of a $\omega$-regular language. Our work is an application framework based 
on the contributions above, with specific focus to schedulability attack analysis. While one 
of the major control objectives discussed in the above contributions has been exponential stability,
which we assume to be satisfied for individual control components, we propose to use a fairness 
schedule in our work, that requires a different modeling strategy, as explained in the following.
More importantly, our work here can be considered as an end-to-end framework for automata-based 
control performance modeling and assessment. \\
  
 
  
