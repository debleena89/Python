\section{Introduction} \label{sec:intro}
\noindent
Most critical infrastructures such as power grid, sewage control, power generation plants, industrial automation etc., are cyber-physical systems. Cyber-Physical systems have the following major interacting constituents -- a set of control components with physical dynamics governed by laws of physics, modeled with partial differential equations, and another control scheduling component, which co-ordinates control of the physical dynamics and schedules them for execution. For very large and geographically distributed cyber physical systems, the control trajectory is designed to be stable and reliable with very complex distributed as well as centralized control. The control components today 
routinely involve a non-trivial amount of software, running on embedded control units on-board, to support and run any moderately sophisticated CPS infrastructure. \\

\noindent
In recent history, cyber physical systems have been a popular victim of choice for so called cyber attacks, the effects of which have been moderate to as ravaging as blackouts~\cite{blackout} or financial loss. This has prompted 
NIST~\cite{nist} to promote cyber security frameworks for critical infrastructure~\cite{cs-Framewrok} as 
one of the themes of immediate research attention. Given that no cyber security framework is full proof, significant research thrust is being invested in recent times to develop techniques that allow us to continually monitor the physical dynamics of these systems and look for any anomalies that could be indicative of cyber attack induced problems. Early detection of system dynamics changes would allow us to contain the damage by immediately islanding parts of the system which point to possible origin areas in the infrastructure. \\

\noindent
Research in security of CPS is 
extremely crucial for developing technologies for cyber attack detection, prevention, and countermeasures. \textcolor{black} {Systematic} studies on different sources of cyber attacks have revealed myriads of possibilities by which these cyber threats can 
propagate inside a CPS infrastructure. 
%A popular and widely acknowledged means of cyber-attack, {\em phishing attacks}, originate when someone whose desktop
%is connected to the control network opens an email containing a payload, which can then
%take over the control of the business network, and in turn the control network. However, since
%much of the cyber threat models also assume insider knowledge or sabotage, even an isolated
%control network is not devoid of cyber-attack possibilities. A popular example  is the notorious Sony Pictures hack~\cite{SonyHack2014} which leaked a release of confidential data from the film studio Sony Pictures. 
A person with local or remote access
to the equipment of the physical plant, or access to the various interfaces such as programmable
logic controllers (PLCs) or other Intelligent Electronic Device (IEDs) that are connected to the
physical system for measurement and control, can exploit a vulnerability in these devices to
induce an attack on the system. One could also gain access to the control network, and create
various kinds of man-in-the-middle attacks by either suppressing measurements or control
actuation signals, replaying stale measurements or actuation signals, or even injecting
maliciously planned false data. These kinds of attacks would then mislead the controllers, and
wrong control actions could lead to disastrous industrial accidents. One could also hack into the
controllers or the various other computing elements in the control center such as the process control servers by
exploiting vulnerabilities in their design and attack the cyber physical system. In fact, in
case of the Stuxnet worm~\cite{stuxnet}, \textcolor{black} {vulnerabilities} in the Siemens SCADA system were exploited. Individual areas of cyber security research have received
much recent attention in academia, e.g. cryptography; crypto-analysis; network
security in the form of firewall and other perimeter security techniques, and intrusion detection, anti-virus software, and static analysis of software to detect vulnerabilities and zero-day attacks;
hardware Trojan detection. \\ 

\noindent
{\bf Problem addressed in this paper:} This paper studies the CPS security problem from a different perspective: the focus of study in this paper is {\em schedulability attacks}, wherein some or parts of a control component are compromised, and the scheduler is unable to schedule the different components in any way to meet the control objectives of the CPS.  In an industrial control environment with real time control components, this can lead to disaster since some of the components may malfunction due to lack of input from the scheduler. As an example, slowing down a
particular process in industrial manufacturing can cascade a chain of failures in the whole
assembly line. In this paper, we propose an idea that will demonstrate that such
attacks for real-time SCADA systems can be guarded against by statically analyzing the
legitimate control programs, and constructing an omega-regular language based timing signature,
which can then be periodically checked on the running components to check for schedulability. The idea presented here is based on the timing signature analysis for omega-regular languages~\cite{WeissFAA09}, in the context of real-time communication scheduling in CPS. \\

\noindent
{\bf Contributions of this work:}
In this paper, we address the problem of detecting schedulability attacks. Given a set of control components, a control objective 
to be satisfied by the control ensemble, the question of schedulability and 
synthesis of a scheduler that can ensure the desired control performance has been recently studied in literature~\cite{WeissFAA09},\cite{AlurW08},~\cite{GhoshMDHD16}. 
In this paper, we extend the same philosophy to build an automata theoretic framework for assessment of replacement attacks on schedulability. The foundation of our attack analysis framework is based on the notion of infinite automata-based reasoning of control performance and schedulability analysis, as illustrated in the following section. Automata theoretic modeling frameworks have been quite popular in literature for a wide variety of applications. Additionally, the power of finite automata over infinite words (B\"{u}chi automatons in particular) have been exploited in recent literature in control performance and stability analysis. Our work is another step in the same direction for assessing the effect of schedulability attacks in cyber-physical control. \\


\noindent
{\bf Organization:} This paper is organized as follows. Section~\ref{sec2} presents related work. Section ~\ref{sec3} describes the problem definition for this work along with a motivating example. Section~\ref{sec4} presents the solution architecture. Section~\ref{sec5} presents an overview of the tool we have developed. Section~\ref{sec7} concludes this discussion.
